\documentclass[]{article}
\usepackage{lmodern}
\usepackage{amssymb,amsmath}
\usepackage{ifxetex,ifluatex}
\usepackage{fixltx2e} % provides \textsubscript
\ifnum 0\ifxetex 1\fi\ifluatex 1\fi=0 % if pdftex
  \usepackage[T1]{fontenc}
  \usepackage[utf8]{inputenc}
\else % if luatex or xelatex
  \ifxetex
    \usepackage{mathspec}
  \else
    \usepackage{fontspec}
  \fi
  \defaultfontfeatures{Ligatures=TeX,Scale=MatchLowercase}
\fi
% use upquote if available, for straight quotes in verbatim environments
\IfFileExists{upquote.sty}{\usepackage{upquote}}{}
% use microtype if available
\IfFileExists{microtype.sty}{%
\usepackage[]{microtype}
\UseMicrotypeSet[protrusion]{basicmath} % disable protrusion for tt fonts
}{}
\PassOptionsToPackage{hyphens}{url} % url is loaded by hyperref
\usepackage[unicode=true]{hyperref}
\hypersetup{
            pdfborder={0 0 0},
            breaklinks=true}
\urlstyle{same}  % don't use monospace font for urls
\usepackage[margin=1in]{geometry}
\usepackage{longtable,booktabs}
% Fix footnotes in tables (requires footnote package)
\IfFileExists{footnote.sty}{\usepackage{footnote}\makesavenoteenv{long table}}{}
\IfFileExists{parskip.sty}{%
\usepackage{parskip}
}{% else
\setlength{\parindent}{0pt}
\setlength{\parskip}{6pt plus 2pt minus 1pt}
}
\setlength{\emergencystretch}{3em}  % prevent overfull lines
\providecommand{\tightlist}{%
  \setlength{\itemsep}{0pt}\setlength{\parskip}{0pt}}
\setcounter{secnumdepth}{0}
% Redefines (sub)paragraphs to behave more like sections
\ifx\paragraph\undefined\else
\let\oldparagraph\paragraph
\renewcommand{\paragraph}[1]{\oldparagraph{#1}\mbox{}}
\fi
\ifx\subparagraph\undefined\else
\let\oldsubparagraph\subparagraph
\renewcommand{\subparagraph}[1]{\oldsubparagraph{#1}\mbox{}}
\fi

% set default figure placement to htbp
\makeatletter
\def\fps@figure{htbp}
\makeatother


\date{}

\begin{document}

\section{PROJECT Design
Documentation}\label{project-design-documentation}

\subsection{Team Information}\label{team-information}

\begin{itemize}
\tightlist
\item
  Team name: Resistance
\item
  Team members
\item
  Justin Lam
\item
  Alan Tan
\item
  Jesse Chen
\item
  Elijah Cantella
\item
  Jay Gogri
\end{itemize}

\subsection{Executive Summary}\label{executive-summary}

This is the documentation for our project after Sprint 1. What has been
implemented is creating the backend for the Sign In and also displaying
the Checker board with the pieces on its respective sides.

\subsubsection{Purpose}\label{purpose}

To allow users to sign in, sign out, and enter a game with another
player by clicking on the opponent's name in the home page.

\subsubsection{Glossary and Acronyms}\label{glossary-and-acronyms}

\begin{quote}
Provide a table of terms and acronyms.
\end{quote}

\begin{longtable}[]{@{}ll@{}}
\toprule
Term & Definition\tabularnewline
\midrule
\endhead
VO & Value Object\tabularnewline
\bottomrule
\end{longtable}

\subsubsection{Requirements}\label{requirements}

\begin{quote}
This section describes the features of the application. In this section
you do not need to be exhaustive and list every story. Focus on
top-level features from the Vision document and maybe Epics and critical
Stories.
\end{quote}

During this sprint, we were given two main tasks to complete * Implement
a Sign In interface so that users can sign in and sign out * Name
restrictions * No duplicate names * No null names such as double quotes
(``) * List out all the users currently online only if the user is sign
in * Implement a Game interface so that users can start a game with each
other * Utliizing that list a user can click on another player and start
a game with him/her * If selected player is in a game than, a message
should show that a game cannot be started * The pieces on the board
should be displayed properly, with challenger as red and challenged as
white, with respective pieces on the bottom of each user's board *
Pieces should be draggable and droppable on valid places on the baord

\subsubsection{Definition of MVP}\label{definition-of-mvp}

The Minimun Viable Product should be a product that can sign a user in
and out (if such user is already signed in), and start a game with
properly aligned pieces.

\subsubsection{MVP Features}\label{mvp-features}

\begin{itemize}
\tightlist
\item
  Sign In

  \begin{itemize}
  \tightlist
  \item
    Player Sign-In
  \item
    Player Sign-Out
  \end{itemize}
\item
  Game Play

  \begin{itemize}
  \tightlist
  \item
    Player Setup
  \item
    Disc Placement
  \end{itemize}
\end{itemize}

\subsubsection{Roadmap of Enhancements}\label{roadmap-of-enhancements}

\begin{quote}
Provide a list of top-level features in the order you plan to consider
them.
\end{quote}

\subsection{Application Domain}\label{application-domain}

This section describes the application domain.

\subsubsection{Overview of Major Domain
Areas}\label{overview-of-major-domain-areas}

\begin{quote}
Provide a high-level overview of the
\end{quote}

\subsubsection{Details of each Domain
Area}\label{details-of-each-domain-area}

\begin{quote}
If necessary, high-light certain areas of the Domain model that have a
focused purpose. Create textual narrative that describes the purpose and
how that relates to the associated domain model.
\end{quote}

\subsection{Architecture}\label{architecture}

This section describes the application architecture.

\subsubsection{Summary}\label{summary}

\begin{quote}
Provide a brief summary of the architecture. Also provide one or two
models (diagrams) that describe the architecture. Hint: review the
Architecture lecture slides for ideas.
\end{quote}

\subsubsection{Overview of User
Interface}\label{overview-of-user-interface}

\begin{quote}
Provide a summary of the application's user interface. This includes the
UI state model.
\end{quote}

\subsubsection{Tier X}\label{tier-x}

\begin{quote}
Provide a summary of each tier of your architecture. Thus replicate this
heading for each tier. In each section describe the types of components
in the tier and describe their responsibilities.
\end{quote}

\subsection{Sub-system X}\label{sub-system-x}

\begin{quote}
Provide a section for each major sub-system within the tiers of the
architecture. Replace `X' with the name of the sub-system. A sub-system
would exist within one of the application tiers and is a group of
components cooperating on a significant purpose within the application.
For example, in WebCheckers all of the UI Controller components for the
Game view would be its own sub-system.
\end{quote}

This section describes the detail design of sub-system X.

\subsubsection{Purpose of the
sub-system}\label{purpose-of-the-sub-system}

\begin{quote}
Provide a summary of the purpose of this sub-system.
\end{quote}

\subsubsection{Static models}\label{static-models}

\begin{quote}
Provide one or more static models (UML class or object diagrams) with
some details such as critical attributes and methods. If the sub-system
is large (over 10 classes) then consider decomposing into multiple,
smaller, more focused diagrams.
\end{quote}

\subsubsection{Dynamic models}\label{dynamic-models}

\begin{quote}
Provide any dynamic model, such as state and sequence diagrams, as is
relevant to a particularly significant user story. For example, in
WebCheckers you might create a sequence diagram of the
\texttt{POST\ /validateMove} HTTP request processing or you might use a
state diagram if the Game component uses a state machine to manage the
game.
\end{quote}

\end{document}
